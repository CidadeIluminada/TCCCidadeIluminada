%% abtex2-modelo-artigo.tex, v-1.9.5 laurocesar
%% Copyright 2012-2015 by abnTeX2 group at http://www.abntex.net.br/
%%
%% This work may be distributed and/or modified under the
%% conditions of the LaTeX Project Public License, either version 1.3
%% of this license or (at your option) any later version.
%% The latest version of this license is in
%%   http://www.latex-project.org/lppl.txt
%% and version 1.3 or later is part of all distributions of LaTeX
%% version 2005/12/01 or later.
%%
%% This work has the LPPL maintenance status `maintained'.
%%
%% The Current Maintainer of this work is the abnTeX2 team, led
%% by Lauro César Araujo. Further information are available on
%% http://www.abntex.net.br/
%%
%% This work consists of the files abntex2-modelo-artigo.tex and
%% abntex2-modelo-references.bib
%%

% ------------------------------------------------------------------------
% ------------------------------------------------------------------------
% abnTeX2: Modelo de Artigo Acadêmico em conformidade com
% ABNT NBR 6022:2003: Informação e documentação - Artigo em publicação
% periódica científica impressa - Apresentação
% ------------------------------------------------------------------------
% ------------------------------------------------------------------------

\documentclass[
	% -- opções da classe memoir --
	article,			% indica que é um artigo acadêmico
	11pt,				% tamanho da fonte
	oneside,			% para impressão apenas no verso. Oposto a twoside
	a4paper,			% tamanho do papel.
    % twocolumn,
	% -- opções da classe abntex2 --
	%chapter=TITLE,		% títulos de capítulos convertidos em letras maiúsculas
	%section=TITLE,		% títulos de seções convertidos em letras maiúsculas
	%subsection=TITLE,	% títulos de subseções convertidos em letras maiúsculas
	%subsubsection=TITLE % títulos de subsubseções convertidos em letras maiúsculas
	% -- opções do pacote babel --
	english,			% idioma adicional para hifenização
	brazil,				% o último idioma é o principal do documento
	sumario=tradicional
	]{abntex2}


% ---
% PACOTES
% ---

% ---
% Pacotes fundamentais
% ---
\usepackage{lmodern}			% Usa a fonte Latin Modern
\usepackage[T1]{fontenc}		% Selecao de codigos de fonte.
\usepackage[utf8]{inputenc}		% Codificacao do documento (conversão automática dos acentos)
\usepackage{indentfirst}		% Indenta o primeiro parágrafo de cada seção.
\usepackage{nomencl} 			% Lista de simbolos
\usepackage{color}				% Controle das cores
\usepackage{graphicx}			% Inclusão de gráficos
\usepackage{microtype} 			% para melhorias de justificação
% ---

% ---
% Pacotes adicionais, usados apenas no âmbito do Modelo Canônico do abnteX2
% ---
\usepackage{lipsum}				% para geração de dummy text
% ---

% ---
% Pacotes de citações
% ---
\usepackage[brazilian,hyperpageref]{backref}	 % Paginas com as citações na bibl
\usepackage[alf]{abntex2cite}	% Citações padrão ABNT
% ---

% ---
% Configurações do pacote backref
% Usado sem a opção hyperpageref de backref
\renewcommand{\backrefpagesname}{Citado na(s) página(s):~}
% Texto padrão antes do número das páginas
\renewcommand{\backref}{}
% Define os textos da citação
\renewcommand*{\backrefalt}[4]{
	\ifcase #1 %
		Nenhuma citação no texto.%
	\or
		Citado na página #2.%
	\else
		Citado #1 vezes nas páginas #2.%
	\fi}%
% ---

% ---
% Informações de dados para CAPA e FOLHA DE ROSTO
% ---
\titulo{Sistema de Gerenciamento de manutenção de pontos de iluminação pública
\\ CidadeIluminada}
\autor{Arthur de Paula Bressan \\(IBTA) \and Thais Matias Nogueira \\(IBTA)}
\local{São José dos Campos, Sâo Paulo, Brasil}
\data{2015}
% ---

% ---
% Configurações de aparência do PDF final

% alterando o aspecto da cor azul
\definecolor{blue}{RGB}{41,5,195}

% informações do PDF
\makeatletter
\hypersetup{
     	%pagebackref=true,
		pdftitle={\@title},
		pdfauthor={\@author},
    	pdfsubject={Modelo de artigo científico com abnTeX2},
	    pdfcreator={LaTeX with abnTeX2},
		pdfkeywords={abnt}{latex}{abntex}{abntex2}{artigo científico},
		colorlinks=true,       		% false: boxed links; true: colored links
    	linkcolor=blue,          	% color of internal links
    	citecolor=blue,        		% color of links to bibliography
    	filecolor=magenta,      		% color of file links
		urlcolor=blue,
		bookmarksdepth=4
}
\makeatother
% ---

% ---
% compila o indice
% ---
\makeindex
% ---

% ---
% Altera as margens padrões
% ---
\setlrmarginsandblock{3cm}{3cm}{*}
\setulmarginsandblock{3cm}{3cm}{*}
\checkandfixthelayout
% ---

% ---
% Espaçamentos entre linhas e parágrafos
% ---

% O tamanho do parágrafo é dado por:
\setlength{\parindent}{1.3cm}

% Controle do espaçamento entre um parágrafo e outro:
\setlength{\parskip}{0.2cm}  % tente também \onelineskip

% Espaçamento simples
\SingleSpacing

% ----
% Início do documento
% ----
\begin{document}

% Seleciona o idioma do documento (conforme pacotes do babel)
%\selectlanguage{english}
\selectlanguage{brazil}

% Retira espaço extra obsoleto entre as frases.
\frenchspacing

% ----------------------------------------------------------
% ELEMENTOS PRÉ-TEXTUAIS
% ----------------------------------------------------------

%---
%
% Se desejar escrever o artigo em duas colunas, descomente a linha abaixo
% e a linha com o texto ``FIM DE ARTIGO EM DUAS COLUNAS''.
% \twocolumn[    		% INICIO DE ARTIGO EM DUAS COLUNAS
%
%---
% página de titulo
\maketitle

% resumo em português
\begin{resumoumacoluna}
    Este trabalho apresenta um estudo sobre dispositivos móveis, explicando a
    implementação de um protótipo de software que auxiliará no recebimento e
    análise de denuncias relacionados a iluminação pública na cidade de
    São José dos Campos. O aplicativo foi desenvolvido em Android e será
    responsável por enviar estas denuncias para um ambiente web.
    A aplicação web será responsável por responder às requisições do aplicativo
    Android e processa-las de modo que a manutenção dos pontos de iluminação
    pública seja feita de uma maneira mais rápida e menos burocrática.
 \vspace{\onelineskip}

 \noindent
 \textbf{Palavras-chave}: Iluminação pública. Aplicativo Android. Sistema web para manutenção.
\end{resumoumacoluna}

% ]  				% FIM DE ARTIGO EM DUAS COLUNAS
% ---

% ----------------------------------------------------------
% ELEMENTOS TEXTUAIS
% ----------------------------------------------------------
\textual

% ----------------------------------------------------------
% Introdução
% ----------------------------------------------------------
\section*{Introdução}
\addcontentsline{toc}{section}{Introdução}

A Iluminação pública é um instrumento de cidadania, pois é um item essencial
à qualidade de vida e a segurança dos centros urbanos, ajudando a preservar e
embelezar os patrimônios públicos e a prática de diversas atividade noturnas
no espaço público.

Segundo \cite{aver} a falta de iluminação contribui consideravelmente com o aumento
dos índices de criminalidade e a falta de segurança dos cidadãos que por algum
motivo tem que transitar no período noturno por pontos sem iluminação, ou com
iluminação precária. Estudos feitos pela organização Campbell Collaboration \cite{newton},
mostraram que a melhoria na iluminação pública em algumas cidades do Reino Unido e EUA, ajudaram a reduzir a criminalidade em até 21%, e concluiu que esta melhoria promoveu o aumento da vigilância da comunidade, que atuou como elemento inibidor contra a criminalidade local.
Os responsáveis pela instalação e manutenção da iluminação pública dos municípios
no Brasil mudou em dezembro de 2014. Até então, elas eram feitas pelo estado,
normalmente através das concessionárias de energia elétrica. Agora é a própria
prefeitura que tem a responsabilidade por garantir as lâmpadas acessas. Essa
mudança foi ocasionada pela Resolução Normativa 414 \cite{aneel414}, uma determinação
constitucional expedida pela ANEEL (Agencia Nacional de Energia Elétrica) que
obrigou todos os 5.570 municípios brasileiros a se responsabilizar pela
manutenção da infraestrutura da iluminação pública.

A cidade de São José dos Campos/SP conta hoje com cerca de 75.000 pontos de iluminação
distribuídos pela cidade. A média de lâmpadas com defeito é cerca de 900 por mês, de
acordo com a Secretaria de Obras da cidade \cite{secretariaobras}. Um dos maiores problemas a
serem enfrentados é o recebimento de chamados para solucionar problemas com os
postes defeituosos, já que o canal de reclamações da prefeitura engloba todos
os problemas da cidade, causando assim, uma sobrecarga no atendimento.

A Urbanizadora Municipal (URBAM) foi a empresa contratada pela prefeitura de
São José dos Campos para realizar a manutenção da iluminação pública do município.
Com a proposta de auxiliar a URBAM no encontro dos pontos com problemas e agilizar
as respostas aos chamados abertos pelos munícipes, o trabalho “Cidade Iluminada”
apresenta uma solução de um aplicativo para smartphones e uma aplicação Web para
a gestão das solicitações de manutenção dos pontos de iluminação pública.

O atual processo de recebimento de chamados, funciona de maneira manual, onde
um funcionário deve entrar em um sistema do canal de recebimentos de denuncias,
o 156, e coletar manualmente todas as denuncias recebidas no dia, colocar em uma
planilha no Excel, e separar por regiões. Após este passo, a planilha já
dividida em regiões, são checados os chamados um a um caso haja duplicidade.
Uma vez que todos os pedidos já estão na planilha e caso não haja duplicidade,
uma ordem de serviço é feita e enviada a URBAM para que a manutenção seja feita.
Quando a manutenção é feita, esta ordem de serviço retorna e são retirados da
planilha os chamados que foram atendidos.

Evidentemente é um processo muito complexo, e passível a erros, onde por um
menor descuido vários detalhes poderão ser perdidos, como denuncias duplicadas
que irão gerar o retrabalho quando a ordem de serviço é gerada. Por isso nós
propomos uma solução, onde não será mais necessária a coleta e verificação
manual de todas as denuncias. Todo este trabalho excessivo será reduzido e em
poucos cliques toda a análise será feita automaticamente.

Nossa missão é disponibilizar uma ferramenta que irá aperfeiçoar o recebimento
dos chamados relacionados aos postes de iluminação e viabilizar a comunicação
do cidadão diretamente com a entidade competente  através de uma aplicação
Android e um ambiente Web onde essas informações serão processadas . Neste
aplicativo, o usuário poderá registrar o problema na rede pública de iluminação,
enviar uma foto e a localização do problema e acompanhar o andamento do
protocolo aberto por ele. No ambiente Web, este protocolo será analisado e
separado de acordo com a região, bairro, rua e tipo de problemas, também serão
analisados se não há duplicidade em algum ponto. E além disso, o sistema tem
como função preparar a ordem de serviço para a manutenção, e fazer o controle
dos materiais utilizados.

% ----------------------------------------------------------
% Seção de explicações
% ----------------------------------------------------------
\section{Fundamentação Teórica}

\subsection{Engenharia de Software}

Segundo Pressman, a engenharia de software é definida como: “O estabelecimento
e uso de sólidos princípios de engenharia para que se possa obter economicamente
um software que seja confiável e que funcione eficientemente em máquinas reais.”\cite{pressman}

O termo engenharia de software foi criado por volta da década de 60, com o
objetivo de sanar a crise do software e contornar os problemas oriundos do
grande crescimento da demanda de softwares e da complexidade a eles imposta,
assim padronizando o desenvolvimento deixando-o mais controlado e sistemático.

Segundo Breitman “o desenvolvimento de sistemas de software é uma
atividade complexa que envolve um grande número de recursos, coordenados de
modo a atingir um mesmo objetivo”. \cite{breitman} Por isso esta tarefa exige um planejamento
detalhado e documentado por se tratar de um objeto dinâmico que está sempre se
adaptando as necessidades dos usuários de acordo com as ferramentas usadas no
seu desenvolvimento

Para Pressman, a engenharia de software tem três fases: definição,
desenvolvimento e manutenção \cite{pressman}.

A fase de definição, estabelece as propriedades, características, limitações e
funcionalidades do software. Esta fase é dividida em três etapas, a análise do
sistema onde são definidas as funções que o sistema deve apresentar.
O planejamento de projeto, onde  são analisados os riscos, custos e recursos que
serão necessários no desenvolvimento. E por fim a análise dos requisitos,
nesta etapa são verificados os requisitos necessários para que atenda as
funções do sistema.

A fase de desenvolvimento, é a parte onde os desenvolvedores transformas as
especificações em código. O projeto é analisado, estabelecendo a arquitetura e
estrutura dos dados, seguida pela codificação numa linguagem de programação
que atenta as necessidades do projeto.

E finalmente a fase de manutenção dedica-se a aprimorar ainda amais o sistema,
sendo validado pelo usuário final, que nem sempre está satisfeito. Algumas
mudanças podem ocorrer no projeto final, como correções pontuais que acontecem
quando usuário acha um erro, adaptações caso novos atributos precisarem ser
adicionados, e por fim o aprimoramento funcional, que acontece a medida que o
usuário se acostuma a utilizar o sistema, e poderá sugerir novas funcionalidades.

\subsection{Método de desenvolvimento Agile, Kanban e Scrum}

O Agile foi publicado em 2001 por meio da publicação de um manifesto por Jeff
Sutherland, Ken Schwaber et. al. que priorizava a rápida adaptação às novas
situações, a colaboração entre si dos times de desenvolvimento, a colaboração dos
times de desenvolvimento com os seus clientes e a entrega de softwares que estejam
funcionando. \cite{agilemanifesto}

Apesar do Agile não ditar como o desenvolvimento deva ser feito por si só, ele
é um conjunto de conceitos para que métodos, como o Kanban e o Scrum, sejam
construídos.

O Scrum consiste basicamente em separar atividades de trabalho em sprints, com
o objetivo de entregar todas as atividades da forma mais completa possível,
trabalhando assim de modo incremental a cada sprint até que o cliente decida que
o software esteja com uma qualidade suficiente para ser entregue. Além disso,
seguindo os princípios dos métodos Agile, o Scrum ajuda na colaboração intra-equipe
por meio de reuniões stand-up diárias, uma reunião pós-sprint de retrospectiva
e uma reunião pré-sprint para planejamento do próximo sprint. Para facilitar o
planejamento, existem somente três “papéis” no Scrum: O Dono do Produto (Product
Owner), o Mestre do Scrum (Scrum Master) e a Equipe (Team).

A Equipe consiste nos desenvolvedores que irão efetivamente fazer a
implementação do produto baseado nas necessidades do Dono, que é responsável por
demonstrar os seus requisitos. O Mestre do Scrum é responsável por além de
transformar os requisitos do Dono em atividades que podem ser divididas entre os
membros da Equipe, é responsável principalmente por gerenciar a equipe para que
as atividades sejam cumpridas em cada sprint. \cite{scrum}

O Kanban é uma técnica que separa as atividades em estados pré-definidos pela
equipe que a utiliza, e é utilizada principalmente para visualizar o progresso
de diferentes atividades dentro de um contexto. Por si só, o Kanban não diz como
introduzir novos processos, mas sim como montar os processos existentes, como o
Scrum, de forma visual.

\subsection{Resolução Normativa 414}

A Agência Nacional de Energia Elétrica (Aneel), determinou que a partir de janeiro
de 2015, a responsabilidade da manutenção da iluminação pública será do município.
Esta resolução vem de uma determinação da Constituição Federal de 1988 que,
em seu artigo 149-A e conforme Emenda Constitucional número 39, de 19 de dezembro
de 2002. A partir desta data limite, o município terá que arcar também com os
investimentos para a ampliação e modernização da rede, instalação de medidores
de acordo com a concessionária e também a manutenção do cadastro de ativos
georreferenciados.

A responsabilidade da manutenção da rede até então era das concessionarias de
distribuição de energia elétrica locais. E segundo um sondagem da Confederação
Nacional dos Municípios (CNM), esse transferência poderá aumentar aproximadamente
até 28\% a despensa anual dos municípios.\cite{cdu}

\subsection{Desenvolvimento mobile}

Durante as últimas décadas foi notável o aumento significativo do uso de
dispositivos móveis, segundo o grupo Ibope[12], com diversas funções e finalidades
 específicas, como computadores de bolso, smartphones, telefones celular,
 consoles portáteis combinado com as mais diversas tecnologias como GPS,
 navegadores de Internet, WAP, leitores de áudio, vídeo e texto, entre outros,
 e juntamente com a evolução das tecnologias e o aumento do poder computacional
 tornou-se possível o desenvolvimento de aplicações independentes do hardware ou
 software do dispositivo.

Na América Latina não foi diferente, até o ano de 2014 o Brasil possuía certa
de 137 milhões de usuário,um pouco mais de 67\% da população, segundo pesquisa
conduzida pela eMarketer [19], que ainda prevê um aumento de mais 4 milhões de
usuários até o final de 2015.

Com base nesses dados é irrefutável o crescimento do mercado neste segmento o
que propicia muitas oportunidades no desenvolvimento de aplicativos mobile.
É visível a necessidade de encontrar novas soluções para os usuários de
smartphones, e integrar novas tenologias a este mercado, pois a necessidade
 não para de crescer junto com  número de usuários.

\section{Tecnologias}

As tecnologias utilizadas foram escolhidas de acordo com o seu nicho específico,
tendo em mente que cada uma delas cumpre um papel importante para entregar uma
solução de qualidade para o cliente.

\subsection{Python}

Python é uma linguagem de programação multiparadigma (se destacando os paradigmas
orientado a objetos e funcional) , interpretada, imperativa e de alto nível;
possui tipagem dinâmica e código fonte livre \cite{pythonlicense} e gratuito, que pode ser
facilmente encontrado na Internet e podendo ser livremente modificado e
redistribuído, e também possui fácil integração com outras plataformas como
.NET, Java e C/C++.

A linguagem foi originalmente concebida e implementada pelo holandês Guido Van
Rossum no início da década de 1990 \cite{pythonlicense}, e a ideia original era de criar um sucessor para
a linguagem ABC.

Uma das grandes vantagens de utilizar o Python é devido à comunidade que surgiu
ao redor da linguagem, ocasionando portanto, em um grande número de bibliotecas
(cerca de 70,000 pacotes, de acordo com o repositório central Python Package
Index)\cite{pypi} tanto open-source como bibliotecas de livre utilização.

O Python foi escolhido para a implementação, junto com o framework web Flask,
devido a sua facilidade e velocidade de implementação, além de experiência
anteriores com o Python e com o framework Flask.

\subsubsection{Flask}

Flask é um microframework para desenvolvimento web, criado em Python baseado
em duas ferramentas Werkzeug, como servidor de aplicação e Jinja2, como o motor
de renderização. O Flask mantém o núcleo simples e, para isso, ele suporta
extensões que são capazes de adicionar diversas funcionalidades a aplicação
como por exemplo suporte a banco de dados, camadas de cache, geração de
formulários, etc.. \cite{flask}

Existem vários outros frameworks de desenvolvimento web para Python, como por
exemplo o Django, Pyramid, Tornado, entre outros.  O Flask foi escolhido devido
a sua grande capacidade de extensão e por causa de experiências anteriores com
o desenvolvimento de aplicativos Web.


\subsubsection{SQLAlchemy e Flask-SQLAlchemy}

O SQLAlchemy é um conjunto de ferramentas para operar com SQL no Python de
forma abstrata, englobando assim um ORM e um conjunto de ferramentas para a
utilização do SQL no Python de forma natural para os usuários de Python. \cite{sqlalchemy}

Além disso, o SQLAlchemy abstrai os principais bancos de dados, como o MySQL,
PostgresSQL, Oracle e até mesmo SQLite. Isso permite a codificação do acesso ao
banco de dados da aplicação sem precisar depender de qual banco específico.

A integração do Flask com o SQLAlchemy se dá por meio da extensão
Flask-SQLAlchemy, e ela tem como objetivo principal facilitar a configuração do
SQLAlchemy para a sua utilização em aplicativos Flask. \cite{flasksqlalchemy}

O SQLAlchemy e o Flask-SQLAlchemy foram escolhidos devido a sua grande
flexibilidade e poder expressivo dentro de sua linguagem, além de sua fácil
integração com o Flask.

\subsubsection{Flask-Migrate e alembic}

O alembic é uma ferramenta que trabalha junto com o SQLAlchemy para a geração e
execução de migrações de banco de dados. Isso permite que alterações feitas no
código das classes sejam refletidas na estrutura de tabelas no banco de dados
de forma semi-automática. \cite{alembic}

O Flask-Migrate é a biblioteca de integração do alembic com o Flask. Isso expõe
ao Flask comandos que facilitam o gerenciamento destas migrações. \cite{flaskmigrate}

O alembic foi escolhido não somente pela sua fácil integração com tanto o
SQLAlchemy e o Flask, mas também por ele ser uma ferramenta de gerenciamento de
migrações de alta qualidade.


\subsubsection{Flask-Admin}

O Flask-Admin é uma extensão do Flask que auxilia a criação de páginas de
gerenciamento do back-end de aplicativos em Flask. Isso é feito expondo aos
programadores utilidades que geram telas administrativas genéricas para que
então cada aplicação as customize de acordo com suas necessidades. \cite{flaskadmin}

O Flask-Admin foi escolhido para ser a base das telas do sistema de
gerenciamento de ordens de serviço pois grande parte das funcionalidades básicas
necessárias para um sistema CRUD já estão maduras no Flask-Admin.

\subsubsection{Flask-Script}

O Flask-Script disponibiliza um conjunto de ferramentas e classes em Python
para o gerenciamento da aplicação Flask via linha de comando.

Inicialmente, o Flask-Script disponibiliza comandos para o gerenciamento do
servidor de desenvolvimento e disponibiliza estruturas para que cada aplicação
possa implementar seus próprios comandos. \cite{flaskscript} Por exemplo a extensão Flask-Migrate
disponibiliza comandos via o Flask-Script para preparar e executar migrações de
banco de dados.

O Flask-Script foi escolhido para realizar a integração pois ela atende
satisfatoriamente à demanda de auxílio no gerenciamento da aplicação.

\subsubsection{xhtml2pdf}

O xhtml2pdf é uma biblioteca que tem como objetivo transformar páginas HTML
preparadas para o formato PDF.\cite{xhtml2pdf}

A biblioteca foi utilizada para gerar as tabelas das ordens de serviço para
compartilhar com a URBAM os endereços dos postes de cada ordem de serviço gerada.

Essa biblioteca foi utilizada pois ela era a mais conveniente de ser utilizada
dado o fato que o Flask contém um renderizador de templates HTML.

\subsubsection{fabric e fabtools}

O fabric é uma utilidade que tem como objetivo executar comandos de
gerenciamento de servidores via SSH. A grande vantagem de sua utilização é
automatizar tarefas de lançamento da aplicação ou até mesmo a preparação de
ambientes de produção.\cite{fabric}

O fabtools é uma biblioteca que expande o fabric para que a escrita de scripts
fabric seja ainda mais fácil. Para isso, o fabtools expõe comandos para as
ferramentas mais utilizadas, como comandos do sistema de arquivos, gerenciamento
de pacotes de instalações, git, banco de dados, etc.. \cite{fabtools}

O fabric e fabtools foram escolhidos para serem utilizados devido experiências
positivas anteriores de gerenciamento de servidores.

\subsubsection{unicodecsv}

O módulo csv do Python é disponibilizado na biblioteca padrão do Python para a
leitura e escrita de arquivos separados por vírgula. Porém, devido às
dificuldades do Python 2.7 para a leitura e uso de textos com codificação em
Unicode, surgiram várias bibliotecas que vieram para suplantar os módulos da
biblioteca padrão.

Uma dessas bibliotecas é justamente o unicodecsv, que simplesmente faz o
tratamento de textos em Unicode para que não ocorra erros na leitura e escrita
desses arquivos com a codificação Unicode. \cite{unicodecsv}

Ela foi escolhida pois, além de atender perfeitamente essa demanda, ela tem
uma grande facilidade no seu uso, pois a sua interface e idêntica ao módulo
csv padrão do Python

\subsubsection{Flask-Security}

O Flask-Security é a integração de uma série de bibliotecas do Flask
relacionadas ao controle de acesso de usuários com o objetivo de facilitar a
implementação desse tipo de sistema. \cite{flasksecurity}

Apesar de sua baixa flexibilidade, o Flask-Security foi escolhido para ser
usado no controle de acesso pois ele atendia plenamente as necessidades do
projeto na questão de gerenciamento de usuários e controle de acesso.


\subsection{Java}

A linguagem de programação Java foi lançada em 1995, e, desde o início de seu
desenvolvimento em 1993 ela foi especificada para ser “independente de plataformas,
porém segura o suficiente para atravessar redes e poderosa o suficiente para
substituir código nativo”. \cite{java}

O Java teve as mais diversas aplicações, sendo usado para applets para a Web,
para programas desktop, para serviços em servidores, para aplicações Web e até
rodando em sistemas embarcados. Essa última aplicação provavelmente
possibilitando a implementação do sistema operacional para smartphones Android.

O Java foi utilizado para a implantação do aplicativo Android do Cidade
Iluminada pois é a linguagem que o sistema operacional Android usa
nativamente para as aplicações, tornando assim o desenvolvimento do aplicativos
de acordo com os padrões de desenvolvimento Android.


\subsubsection{Android}

O sistema operacional Android foi lançado pela Google no ano de 2007, e no ano
de 2008 foi lançado uma API para os desenvolvedores. Desde o seu lançamento,
este sistema operacional obteve grande popularidade em seu uso, chegando a ter
mais de 89\% do mercado brasileiro no primeiro trimestre de 2015, segundo o site
Kantar Worldpanel ComTech. [4]

A Google provê uma API de fácil manipulação, com código fonte aberto baseada em
Java para incentivar e simplificar o desenvolvimento de aplicações, o que tem
sido uma boa estratégia já que o número de aplicativos disponíveis no Android
Market chegou a 1.6 milhões no último semestre de 2015.

O sucesso do Android no mercado brasileiro, que segundo a pesquisa do site The
Statistics Portal, o market share do sistema operacional chegou a mais de
77\% no primeiro semestre de 2015[18], e a facilidade de desenvolvimento foram
os principais motivos da escolha do uso deste para o desenvolvimento do
aplicativo Cidade Iluminada.

\subsubsection{GSON}

GSON é uma biblioteca do Google para a serialização e deserialização de objetos
Java arbitrários. O objetivo desta biblioteca, além de prover essas conversões,
é trabalhar com objetos que o programador não necessariamente tem o código-fonte
disponível, mantendo uma interface simples (métodos fromJson e toJson) para
quando for precisar converter de e para JSON. [33]

A biblioteca foi utilizada pois ela atende a necessidade principal de tratar
JSON mas ainda sendo relativamente fácil de utilizar e configurar as classes
que precisam dessa conversão.

\subsubsection{Joda-Time}

A biblioteca Joda-Time tem como objetivo prover facilidades para o programador Java
manipular objetos de data e tempo, tendo em vista que a biblioteca padrão do Java
deixa a desejar no aspecto de facilidade de uso.

Ela foi utilizada para a geração de timestamps em UTC para marcar o envio das
requisições do aplicativo e para gerar os nomes de arquivos das fotos tiradas
pelo aplicativo. A biblioteca Joda-Time foi escolhida para substituir a
biblioteca padrão do Java nas suas proximas versões.[34]

\subsubsection{Retrofit}

A biblioteca Retrofit tem como objetivo disponibilizar ferramentas para que se
possa definir APIs REST como interfaces Java para serem então usadas no código
da aplicação sem maiores dificuldades. Além disso, ela disponibiliza modos de
fazer requisições síncronas ou assíncronas, dependendo das necessidades de cada
aplicação.[35]

A biblioteca Retrofit foi utilizada devido a sua facilidade de utilização e
adaptação à API já definida do trabalho. Houve uma primeira tentativa de
implementação utilizando a biblioteca Unirest, mas devido a dificuldades de
compilação da biblioteca para utilização no ecossistema Android a Unirest foi
substituida pela Retrofit.[36]

\subsubsection{Sugar}

Sugar é uma biblioteca que tem como objetivo de ser uma ORM especificamente para
aplicações Android, tendo uma interface por objetos e classes com o banco de dados
SQLite do dispositivo. Além disso, ela tem a pretensão de ser simples de usar e
simples de configurar. Alem disso, ela possui suporte à migraçoes de banco
automáticas.[37]

O Sugar foi escolhido justamente devido a sua simplicidade tanto na configuração
quanto no seu uso dentro do codigo.

\subsection{JavaScript}

JavaScript é uma linguagem dinâmica, funcional e orientada a objetos, suas
aplicações são distribuídas somente em forma de código fonte, e é executada em
todos os navegadores modernos com poucas ou nenhuma diferença de plataforma.
Isso é especialmente importante nos dias atuais pois até mesmo smartphones têm
o seu próprio navegador, abrindo então uma nova plataforma inteira de aplicações

Em 1994, foi fundada uma empresa chamada Netscape, e um dos principais objetivos
dela foi a exploração da Web nascente, e sua então nova necessidade de serem
disponibilizadas páginas dinâmicas. Um exemplo disso foi a necessidade de
validação de formulários: O usuário teria que preencher todo o formulário,
enviá-lo e então receber o retorno do servidor se falhou ou não. Em 1995, em uma
parceria com a Sun, a Netscape decidiu que esta nova linguagem teria que ter uma
sintaxe parecida com a liguagem Java, descartando assim a liguagem Python
(entre outras) como candidatas para ser implementadas em seu navegador.

Brendan Eich foi contratado para implementar a liguagem Scheme no navegador,
porém, por causa dessa decisão da Netscape, ele acabou implementando a liguagem
que seria conhecida como JavaScript (o nome oficial é ECMAScript) em 10 dias. [8]

Apesar de seu surgimento rápido, a linguagem foi passando por várias alterações
que a deixam muito mais poderosa e mais em linha com as necessidades atuais de
desenvolvimento para a Web moderna.

O JavaScript foi escolhido para a implementação de páginas Web dinâmicas no
projeto principalmente pelo fator de ele ser suportado pelos principais
navegadores existentes no mercado, como o Mozilla Firefox, o Google Chrome e o
Internet Explorer, sendo os dois primeiros softwares livres para a utilização em
ambientes pessoais ou corporativos.


\subsubsection{JSON}

JSON (JavaScript Object Notation) é um formato para guardar e transmitir dados
em formato de texto . JSON descoberto e formalizado por Douglas Crockford em
2001. [37] Desde então vem ganhando popularidade como um formato de troca de
dados entre aplicações web. Existem bibliotecas para a leitura do formato JSON
para a maioria das linguagens de programação, além do suporte praticamente
nativo no JavaScript e Python.

O JSON foi escolhido como método de troca de dados pois ele tem suporte
praticamente nativo em JavaScript e Python, além de ser mais eficiente em termos
de quantidade de bytes enviados do que XML para envio de dados mais simples.

\subsubsection{AngularJS}

AngularJS é um framework do tipo MVC (Model-View-Controller) JavaScript que
tem como objetivo facilitar o desenvolvimento web, estendendo e adaptando o
HTML para melhorar a experiência de desenvolvimento de páginas Web altamente
dinâmicas. Esta framework foi criada em 2009[11] por  Miško Hevery e Adam
Abrons, e é open-source.

O framework foi utilizado na criação da página de controle dos protocolos
vindos da aplicação Android, sendo parte fundamental da implementação do módulo
de tempo real desta página.

O AngularJS foi escolhido devido a sua grande capacidade de facilitar o
desenvolvimento e manutenção de páginas Web altamente interativos.

\subsubsection{JQuery}

JQuery é uma biblioteca JavaScript que tem como objetivo principal facilitar a 4
seleção e manipulação de documentos HTML e o controle de eventos nas páginas.
Isso possibilita a criação de páginas interativas em JavaScript com muito menos
dificuldades. Além disso, o JQuery disponibiliza uma API única para todos os
navegadores principais. [38]

O JQuery foi utilizado em várias das páginas da parte de gerenciamento de
protocolos e serviços, pois ela se tornou uma peça fundamental para o
desenvolvimento Web moderno. Das páginas utilizadas, destaca-se a página de
criação de Ordens de Serviço, onde a parte de seleção de postes foi
implementada em JQuery.


\subsection{Representational State Transfer (REST)}

O REST é um estilo de arquitetura para aplicações Web proposta por Roy Thomas
Fielding em sua dissertação de PhD. [39] Esta tese tinha como objetivo de,
além de propor esse tipo de arquitetura, avaliar se serviços disponíveis na Web
até então seguiam os quesitos seguintes:

\begin{itemize}
\item \textbf{Cliente-Servidor:} A aplicação teria que manter independentemente suas
implementação de clientes (interface gráfica, por exemplo) da implementação dos
serviços oferecidos por ela (banco de dados, por exemplo)

\item \textbf{Sem estado:} Isso significa que, a cada requisição enviada para o servidor,
esta teria que incluir todos os dados para que a própria seja entendida,
sem que o servidor precise depender de informações que por ventura ele possa
ter armazenado anteriormente.

\item \textbf{Cacheável:} Cada resposta a uma requisição tem que deixar explícito se é
permitido guardar a resposta em um cache no cliente. A ideia é que, se o
cliente precisar refazer uma requisição equivalente, esta requisição deveria
utilizar o cache ao invés de ser enviada para o servidor.

\item \textbf{Interface uniforme:} De acordo com Fielding em sua dissertação, é a
característica fundamental que aplicações REST têm de implementar. Ela
consiste em padronizar todos os componentes para que as implementações de
clientes e serviços sejam simplificados.

\item \textbf{Sistema em camadas:} Cada componente deve não conhecer ou depender de
componentes de que este componente não precise interagir. A principal vantagem,
de acordo com Fielding, é que o sistema em camadas pode ser usado para
encapsular clientes e serviços legados

\item \textbf{“Code-on-demand”:} A aplicação REST teria que oferecer um conjunto de recursos
que viriam como código a ser executado no cliente. Este código seria como uma
interface mínima da aplicação para o cliente consumir. A principal vantagem de
implementar o Code-on-demand seria que, a parte básica da API não precisaria
ser reimplementada pelo cliente. É o único quesito opcional para que uma
interface fosse considerada como REST.
\end{itemize}

\section{Desenvolvimento}

O desenvolvimento do trabalho pode ser separado em duas fases ou períodos
distintos, sendo estes o desenvolvimento do aplicativo para smartphones Android
e o sistema de gerenciamento dos protocolos deste, e o desenvolvimento do
sistema de gerenciamento de pontos de iluminação e de seus serviços e chamados.


\subsection{Aplicativo Android e seu sistema de gerenciamento}

Para o desenvolvimento do aplicativo para smartphones Android e seu sistema de
gerenciamento, foram aplicadas os métodos Agile de desenvolvimento Scrum e Kanban.

As atividades a serem cumpridas foram determinadas no início do projeto, e
foram divididas em três categorias, e priorizadas dependendo da importância de
cada atividade para o cumprimento do projeto.

Em relação à execução das atividades, elas foram divididas em 5 “sprints”, que
foi um período de 14 dias em que as atividades selecionadas para fazer parte
do sprint teriam que ser cumpridas. Para determinar quais atividades entrariam
em cada sprint, foi usado primeiramente o critério da prioridade de cada
atividade, e secundariamente, o tempo estimado para o término da atividade.


\subsubsection{Backlog de atividades}

O backlog de atividades é o repositório inicial das atividades, e onde cada
atividade foi posta em uma das três categorias de desenvolvimento, App,
Webservice e Site Secretaria. Além disso, cada atividade recebeu uma pontuação
de 1 a 15, que determinava a prioridade daquela atividade. 1 é a atividade
mais prioritária.

\begin{table}[htbp]
    \caption{Tabela de definição das atividades.}
    \label{tabela-atividades}
    \begin{center}
        \begin{tabular}{|p{5cm}|p{4cm}|p{5cm}|}
            \hline
            8 - Tela Principal & 7 - Incluir protocolos & 1 - Criar Banco de Dados \\
            9 - Incluir protocolos anônimos & 13 - Consultar protocolos & 2 - Página inicial \\
            10 - Cadastro de usuário & 14 - Listar protocolos & 3 - Cadastro de usuário \\
            11 - Incluir protocolos cadastrados & & 4 - Controle de acesso \\
            12 - Gravar off-line protocolos para envio posterior & & 5 - Visualizar protocolos \\
            15 - Listar protocolos & & 6 - Mudar status protocolos \\
            \hline
        \end{tabular}
    \end{center}
\end{table}

\subsubsection{Definição dos Sprints}

Um sprint é um período previamente acordado pelo time Agile onde o sistema é de
fato implementado. Este período pode variar de acordo com cada equipe, mesmo em
equipes dentro da mesma organização ou empresa.

Para o desenvolvimento do sistema de gerenciamento dos protocolos e do aplicativo
Android, cada sprint teve duração de 14 dias, onde foram escolhidos várias
atividades de acordo com os critérios da sua prioridade de implementação e sua
facilidade de implementação dada a situação que o projeto se encontrava.

\begin{description}
\item[Sprint I:]
O primeiro sprint foi feito a modelagem e a implementação do
banco de dados, assim como o layout e a wireframe do ambiente web.

\item[Sprint II:]
Nesta segunda etapa foi feita a criação do layout e padronização da interface
do ambiente web. Também foram implementados os serviços de criação e inclusão
dos protocolos.

\item[Sprint III:]
Implementação das telas do aplicativo Android, e do fluxo principal de telas,
também como a funcionalidade de ativar a câmera do aparelho, também foi
implementado uma funcionalidade que busca automaticamente a localização do
aparelho que está enviando o reporte

\item[Sprint IV:]
Este sprint foi focado na implementação das funcionalidades do aplicativo como
o cadastro do usuário, tela de configurações e status dos protocolos enviados

\item[Sprint V:]
No último sprint foram acertado os últimos detalhes do aplicativo, como
gravação dos protocolos off-line para envio posterior, busca do status do
protocolo e foi finalizado o design da aplicação.
\end{description}

\subsubsection{Apresentação para a Secretaria de Obras}

Após a finalização da primeira versão do aplicativo e do ambiente web do
projeto “Cidade Iluminada”, fomos convidados a apresentar nossa solução para o
Secretário de Obras de São José dos Campos, o senhor Rene Mina Vernice e sua
equipe, na data de 14 de Setembro de 2015.

Inicialmente, a ideia do aplicativo não foi bem aceita, pois foi alegado que
isso geraria ainda mais trabalho para os funcionários responsáveis por
recolher as denuncias relacionadas a iluminação pública. Apesar disso, o
conceito do ambiente web para gerenciamento foi bem recebido, pois eles
enfrentam muitas dificuldades administrando os chamados já existente.

\subsection{Sistema de gerenciamento de pontos de iluminação de ordens de serviço}

Como segunda fase do trabalho, foi definido que seria desenvolvido um ambiente
onde os funcionários da Secretaria de Obras pudessem fazer o gerenciamento de
protocolos e ordens de serviço.

\subsubsection{Entrevista inicial ao cliente}

Anteriormente a primeira entrevista, foi realizada uma apresentação do conceito
do projeto para o Secretário de Obras Renê Vernice e outros integrantes da
equipe. O conceito do projeto foi não bem aceito por todos, pois a Secretaria
de Obras não atenderia os chamados abertos pelo aplicativo. Foi acordado que o
projeto seguiria, caso fossem feitas algumas modificações que ajudassem na
organização dos serviços realizados nos pontos de iluminação.

Posteriormente, foi feita uma reunião com a assistente do Secretário, a senhora
Cintia Firmino para entendermos melhor o funcionamento do processo atual.
\begin{enumerate}

\item \textbf{Como é o processo atual?}

Processo de atualização feito a partir da coleta manual do sistema 156, e
colocado numa planilha para o tratamento de pedidos duplicados e separação
dos casos por zona e bairro.

\item \textbf{Como vocês recebem as chamadas dos postes com defeitos?}

Os chamados vêm do sistema 156, ou por memorando de Vereadores que tem
prioridade, não possui um canal personalizado apenas para eles vem do
sistema 156 (156 possui resposta programada)

\item \textbf{Como é feita a ordem de serviço?}

A ordem de serviço é feita pelo Excel, separando os chamados por zona,
bairro e rua. E após isso esta planilha é enviada para a empresa URBAM por
e-mail.

\item \textbf{Vocês acham que é o melhor processo? Como poderia melhorar?}

O processo poderia melhorar se os protocolos já chegassem a nós separados pela
região, sem duplicidade, com a data, sem a necessidade de analisar todos os
pedidos que vem do 156

\item \textbf{Qual seria o processo ideal?}

Receber os protocolos separados pela região, bairro e rua, sem duplicidade.
Responder ao munícipe automaticamente.

\item \textbf{Vocês separam os postes por tipos? Quais tipos?}

Temos dois tipos, a iluminação ornamental, e a iluminação de vias.

\item \textbf{Como vocês agrupam postes da mesma rua? Vocês têm uma base de dados de
endereços?}

Os postes são agrupados em até 10 números da mesma rua. Há problemas
também em verificar os endereços das praças e vielas pois tem mais de um
endereço, além de algumas praças e vielas não constarem na pesquisa do Google.
Isso afeta tamém a troca de làmpadas ornamentais.

\item \textbf{Vocês agrupam as situações das pendências (resolvida, não resolvida,
enviada para serviço, etc.)? Se não, como seria o ideal?}

Sim, por planilha, quando uma ordem de serviço e é mandada, os pedidos são
retirados de uma planilha e colocadas em outra sinalizando que foram mandados
e estamos esperando a resposta da URBAM.

\item \textbf{Vocês agrupam os tipos de serviços das pendências (troca de lâmpada,
troca de reator, etc.)? Se não, como seria o ideal?}

Existem 3 tipos de serviços que nós justo com a URBAM, a troca de lâmpada, a
troca de reator, e a troca da fotocélula.

O que será trocado é analisado um a um, pela URBAM que depois reporta o que
foi trocado.

\item \textbf{Como é feita a identificação de cada poste dentro de uma rua? Pelo
número aproximado da casa?}

Ainda não existe um sistema de identificação no momento.

\item \textbf{Existe algum código para identificar os postes? Se não, o código ideal
deve informar o quê?}

Não existe, é pelo número da casa mais próxima ao poste.

\item \textbf{Vocês atualizam informações sobre cada ocorrência após o retorno da
ordem de serviço? Se sim, o que vocês atualizam?}

Sim, a URBAM retorna uma planilha com as tarefas executadas e o que foi
trocado, e após, estas informações são colocadas numa planilha chamada
“Executados”, e depois é respondido para o munícipe no 156.
\end{enumerate}

\subsubsection{Requisitos do sistema}

Quando pensamos em qualidade de software, o foco está em entender os requisitos
estabelecidos para assim evitar, ou minimizar a insatisfação do cliente final.
Segundo Carvalho, Tavares [14] “A demanda por qualidade tem estimulado a
comunidade de software para o desenvolvimento de modelos que conduzam a
qualidade dos sistemas. Existe uma forte ligação entre requisitos e qualidade.”

Em um sistema computacional, os requisitos definem o escopo e os serviços que o
sistema deve oferecer, assim como as restrições que são aplicáveis a suas
operações.

De acordo com as respostas levantadas na reunião com a Cíntia, foram então levantados os seguintes requisitos funcionais:

\begin{description}
\item[RF1:] O sistema deverá fazer a coleta automática dos protocolos no sistema do 156
direcionados para a secretaria de obras.
\item[RF2:] O sistema deverá fazer essa coleta e, de acordo com a informação de cada
protocolo, separar em bairros e regiões corretas para cada endereço.
\item[RF3:] O sistema deverá fazer a coleta sem adicionar protocolos em duplicidade.
Dois ou mais protocolos em duplicidade se caracterizam por se referirem ao
mesmo ponto de iluminação pública, tendo em vista que os postes são agrupados
logicamente de 10 em 10 metros na rua. Por exemplo, se um protocolo se referir
a um poste no número 10 da rua, e outro protocolo se referir a um poste no número
12 da rua, isso seria o mesmo poste.
\item[RF4:] O sistema deverá tentar ligar a cada protocolo a um poste, caso isso falhe,
os sistemas devem permitir que o usuário faça essa correção manualmente.
\item[RF5:] O sistema deverá gerar a ordem de serviço em formato de planilha com,
no máximo 50 pendências. Na geração da ordem de serviço, o sistema deverá
dar preferência em agrupar protocolos do mesmo bairro e da mesma região da cidade.
\item[RF6:] O sistema deverá permitir a atualização da situação de execução e qual foi
o tipo de manutenção realizada de cada protocolo das ordens de serviço.
\item[RF7:] O sistema deverá identificar as ordens de serviço que foram executadas para
que o munícipe seja alertado.
\item[RF8:] O sistema deverá permitir que se faça buscas na base de postes e protocolos.
\end{description}

% \begin{verbatim}
%    \setlrmarginsandblock{3cm}{3cm}{*}
%    \setulmarginsandblock{3cm}{3cm}{*}
%    \checkandfixthelayout
% \end{verbatim}

% \subsection{Duas colunas}

% É comum que artigos científicos sejam escritos em duas colunas. Para isso,
% adicione a opção \texttt{twocolumn} à classe do documento, como no exemplo:

% \begin{verbatim}
%    \documentclass[article,11pt,oneside,a4paper,twocolumn]{abntex2}
% \end{verbatim}

% É possível indicar pontos do texto que se deseja manter em apenas uma coluna,
% geralmente o título e os resumos. Os resumos em única coluna em documentos com
% a opção \texttt{twocolumn} devem ser escritos no ambiente
% \texttt{resumoumacoluna}:

% \begin{verbatim}
%    \twocolumn[              % INICIO DE ARTIGO EM DUAS COLUNAS

%      \maketitle             % pagina de titulo

%      \renewcommand{\resumoname}{Nome do resumo}
%      \begin{resumoumacoluna}
%         Texto do resumo.

%         \vspace{\onelineskip}

%         \noindent
%         \textbf{Palavras-chave}: latex. abntex. editoração de texto.
%      \end{resumoumacoluna}

%    ]                        % FIM DE ARTIGO EM DUAS COLUNAS
% \end{verbatim}

% \subsection{Recuo do ambiente \texttt{citacao}}

% Na produção de artigos (opção \texttt{article}), pode ser útil alterar o recuo
% do ambiente \texttt{citacao}. Nesse caso, utilize o comando:

% \begin{verbatim}
%    \setlength{\ABNTEXcitacaorecuo}{1.8cm}
% \end{verbatim}

% Quando um documento é produzido com a opção \texttt{twocolumn}, a classe
% \textsf{abntex2} automaticamente altera o recuo padrão de 4 cm, definido pela
% ABNT NBR 10520:2002 seção 5.3, para 1.8 cm.

% \section{Cabeçalhos e rodapés customizados}

% Diferentes estilos de cabeçalhos e rodapés podem ser criados usando os
% recursos padrões do \textsf{memoir}.

% Um estilo próprio de cabeçalhos e rodapés pode ser diferente para páginas pares
% e ímpares. Observe que a diferenciação entre páginas pares e ímpares só é
% utilizada se a opção \texttt{twoside} da classe \textsf{abntex2} for utilizado.
% Caso contrário, apenas o cabeçalho padrão da página par (\emph{even}) é usado.

% Veja o exemplo abaixo cria um estilo chamado \texttt{meuestilo}. O código deve
% ser inserido no preâmbulo do documento.

% \begin{verbatim}
% %%criar um novo estilo de cabeçalhos e rodapés
% \makepagestyle{meuestilo}
%   %%cabeçalhos
%   \makeevenhead{meuestilo} %%pagina par
%      {topo par à esquerda}
%      {centro \thepage}
%      {direita}
%   \makeoddhead{meuestilo} %%pagina ímpar ou com oneside
%      {topo ímpar/oneside à esquerda}
%      {centro\thepage}
%      {direita}
%   \makeheadrule{meuestilo}{\textwidth}{\normalrulethickness} %linha
%   %% rodapé
%   \makeevenfoot{meuestilo}
%      {rodapé par à esquerda} %%pagina par
%      {centro \thepage}
%      {direita}
%   \makeoddfoot{meuestilo} %%pagina ímpar ou com oneside
%      {rodapé ímpar/onside à esquerda}
%      {centro \thepage}
%      {direita}
% \end{verbatim}

% Para usar o estilo criado, use o comando abaixo imediatamente após um dos
% comandos de divisão do documento. Por exemplo:

% \begin{verbatim}
%    \begin{document}
%      %%usar o estilo criado na primeira página do artigo:
%      \pretextual
%      \pagestyle{meuestilo}

%      \maketitle
%      ...

%      %%usar o estilo criado nas páginas textuais
%      \textual
%      \pagestyle{meuestilo}

%      \chapter{Novo capítulo}
%      ...
%    \end{document}
% \end{verbatim}

% Outras informações sobre cabeçalhos e rodapés estão disponíveis na seção 7.3 do
% manual do \textsf{memoir} \cite{memoir}.

% \section{Mais exemplos no Modelo Canônico de Trabalhos Acadêmicos}

% Este modelo de artigo é limitado em número de exemplos de comandos, pois são
% apresentados exclusivamente comandos diretamente relacionados com a produção de
% artigos.

% Para exemplos adicionais de \abnTeX\ e \LaTeX, como inclusão de figuras,
% fórmulas matemáticas, citações, e outros, consulte o documento
% \citeonline{abntex2modelo}.

% \section{Consulte o manual da classe \textsf{abntex2}}

% Consulte o manual da classe \textsf{abntex2} \cite{abntex2classe} para uma
% referência completa das macros e ambientes disponíveis.

% ---
% Finaliza a parte no bookmark do PDF, para que se inicie o bookmark na raiz
% ---
\bookmarksetup{startatroot}%
% ---

% ---
% Conclusão
% ---
\section*{Considerações finais}
\addcontentsline{toc}{section}{Considerações finais}

A gestão de pendencias relativas a iluminação pública da prefeitura de São José
dos Campos é burocrática, o que reflete na lentidão do atendimento e manutenção
de um serviço tão essencial ao bem estar e segurança da população. O fato da
responsabilidade pela manutenção ter sido passada para as prefeituras recentemente
também é um fator crítico no desempenho do atendimento, já que as prefeituras
tiveram que se adaptar rapidamente.

Partindo desta problemática, incentivados pela ideia de agilizar este processo,
desenvolvemos o projeto Cidade Iluminada, onde o ambiente web, que que passou
pela validação da equipe da Secretaria de Obras da Prefeitura Municipal de São
José dos Campos provou reduzir a carga de trabalho e facilitar o recolhimento e
análise das denuncias recebidas, automatizando um trabalho que era feito manualmente.

Infelizmente o aplicativo ainda não será disponibilizado para os munícipes a
pedido do cliente, que alega que este canal de denuncias poderia  neste primeiro
momento congestionar os recebimentos de denuncias já que a equipe ainda não
possui um processo estruturado para a demanda, o aplicativo será usado por
funcionários para a identificação de pontos de iluminação defeituosos.

Como trabalho futuro, foi proposto que o aplicativo seja liberado para o uso
dos munícipes, integrando ao ambiente web para que ele receba as denúncias
diretamente. Para que o aplicativo seja liberado, também é necessário que
além de Android seja desenvolvido para iOS assim alcançando todos os públicos.

Relacionado ao trabalho em questão, seria interessante que futuramente novas
funcionalidades sejam integradas, utilizando o aplicativo para outras
utilizações além de denuncias relacionadas a iluminação pública.

Uma outra possibilidade é a melhoria da interface gráfica, tornando o ambiente
web mais amigável para o uso de funcionários com vários níveis de conhecimento.


% \lipsum[1]

% \begin{citacao}
% \lipsum[2]
% \end{citacao}

% \lipsum[3]

% ----------------------------------------------------------
% ELEMENTOS PÓS-TEXTUAIS
% ----------------------------------------------------------
\postextual

% ---
% Título e resumo em língua estrangeira
% ---

% \twocolumn[    		% INICIO DE ARTIGO EM DUAS COLUNAS

% titulo em inglês
\titulo{Public lighting maintenance management system \\ CidadeIluminada}
\emptythanks
\maketitle

% resumo em português
\renewcommand{\resumoname}{Abstract}
\begin{resumoumacoluna}
 \begin{otherlanguage*}{english}
   This paper presents a study on mobile development, explaining the
   implementation of a software prototype that will assist in receiving
   complaints and analysis related to street lighting in the city of São José
   dos Campos. The application was developed on Android and will be responsible
   for sending these complaints to a web environment. The web application
   will be responsible for responding to requests from the Android application
   and process them so that the maintenance of public lighting points is done
   in a faster and less bureaucratic way.
   \vspace{\onelineskip}

   \noindent
   \textbf{Keywords}: Public lighting system. Android Application. Web system for maintenance.
 \end{otherlanguage*}
\end{resumoumacoluna}

% ]  				% FIM DE ARTIGO EM DUAS COLUNAS
% ---

% ----------------------------------------------------------
% Referências bibliográficas
% ----------------------------------------------------------
\bibliography{bibliografia}

% ----------------------------------------------------------
% Glossário
% ----------------------------------------------------------
%
% Há diversas soluções prontas para glossário em LaTeX.
% Consulte o manual do abnTeX2 para obter sugestões.
%
%\glossary

% ----------------------------------------------------------
% Apêndices
% ----------------------------------------------------------

% ---
% Inicia os apêndices
% ---
% \begin{apendicesenv}

% % ----------------------------------------------------------
% \chapter{Nullam elementum urna vel imperdiet sodales elit ipsum pharetra ligula
% ac pretium ante justo a nulla curabitur tristique arcu eu metus}
% % ----------------------------------------------------------
% \lipsum[55-57]

% \end{apendicesenv}
% ---

% ----------------------------------------------------------
% Anexos
% ----------------------------------------------------------
% \cftinserthook{toc}{AAA}
% % ---
% % Inicia os anexos
% % ---
% %\anexos
% \begin{anexosenv}

% % ---
% \chapter{Cras non urna sed feugiat cum sociis natoque penatibus et magnis dis
% parturient montes nascetur ridiculus mus}
% % ---

% \lipsum[31]

% \end{anexosenv}

\end{document}
