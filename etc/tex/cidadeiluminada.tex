%% abtex2-modelo-artigo.tex, v-1.9.5 laurocesar
%% Copyright 2012-2015 by abnTeX2 group at http://www.abntex.net.br/
%%
%% This work may be distributed and/or modified under the
%% conditions of the LaTeX Project Public License, either version 1.3
%% of this license or (at your option) any later version.
%% The latest version of this license is in
%%   http://www.latex-project.org/lppl.txt
%% and version 1.3 or later is part of all distributions of LaTeX
%% version 2005/12/01 or later.
%%
%% This work has the LPPL maintenance status `maintained'.
%%
%% The Current Maintainer of this work is the abnTeX2 team, led
%% by Lauro César Araujo. Further information are available on
%% http://www.abntex.net.br/
%%
%% This work consists of the files abntex2-modelo-artigo.tex and
%% abntex2-modelo-references.bib
%%

% ------------------------------------------------------------------------
% ------------------------------------------------------------------------
% abnTeX2: Modelo de Artigo Acadêmico em conformidade com
% ABNT NBR 6022:2003: Informação e documentação - Artigo em publicação
% periódica científica impressa - Apresentação
% ------------------------------------------------------------------------
% ------------------------------------------------------------------------

\documentclass[
	% -- opções da classe memoir --
	article,			% indica que é um artigo acadêmico
	11pt,				% tamanho da fonte
	oneside,			% para impressão apenas no verso. Oposto a twoside
	a4paper,			% tamanho do papel.
    twocolumn,
	% -- opções da classe abntex2 --
	%chapter=TITLE,		% títulos de capítulos convertidos em letras maiúsculas
	%section=TITLE,		% títulos de seções convertidos em letras maiúsculas
	%subsection=TITLE,	% títulos de subseções convertidos em letras maiúsculas
	%subsubsection=TITLE % títulos de subsubseções convertidos em letras maiúsculas
	% -- opções do pacote babel --
	english,			% idioma adicional para hifenização
	brazil,				% o último idioma é o principal do documento
	sumario=tradicional
	]{abntex2}


% ---
% PACOTES
% ---

% ---
% Pacotes fundamentais
% ---
\usepackage{lmodern}			% Usa a fonte Latin Modern
\usepackage[T1]{fontenc}		% Selecao de codigos de fonte.
\usepackage[utf8]{inputenc}		% Codificacao do documento (conversão automática dos acentos)
\usepackage{indentfirst}		% Indenta o primeiro parágrafo de cada seção.
\usepackage{nomencl} 			% Lista de simbolos
\usepackage{color}				% Controle das cores
\usepackage{graphicx}			% Inclusão de gráficos
\usepackage{microtype} 			% para melhorias de justificação
% ---

% ---
% Pacotes adicionais, usados apenas no âmbito do Modelo Canônico do abnteX2
% ---
\usepackage{lipsum}				% para geração de dummy text
% ---

% ---
% Pacotes de citações
% ---
\usepackage[brazilian,hyperpageref]{backref}	 % Paginas com as citações na bibl
\usepackage[alf]{abntex2cite}	% Citações padrão ABNT
% ---

% ---
% Configurações do pacote backref
% Usado sem a opção hyperpageref de backref
\renewcommand{\backrefpagesname}{Citado na(s) página(s):~}
% Texto padrão antes do número das páginas
\renewcommand{\backref}{}
% Define os textos da citação
\renewcommand*{\backrefalt}[4]{
	\ifcase #1 %
		Nenhuma citação no texto.%
	\or
		Citado na página #2.%
	\else
		Citado #1 vezes nas páginas #2.%
	\fi}%
% ---

% ---
% Informações de dados para CAPA e FOLHA DE ROSTO
% ---
\titulo{Sistema de Gerenciamento de manutenção de pontos de iluminação pública
\\ CidadeIluminada}
\autor{Arthur de Paula Bressan \\(IBTA) \and Thais Matias Nogueira \\(IBTA)}
\local{São José dos Campos, Sâo Paulo, Brasil}
\data{2015}
% ---

% ---
% Configurações de aparência do PDF final

% alterando o aspecto da cor azul
\definecolor{blue}{RGB}{41,5,195}

% informações do PDF
\makeatletter
\hypersetup{
     	%pagebackref=true,
		pdftitle={\@title},
		pdfauthor={\@author},
    	pdfsubject={Modelo de artigo científico com abnTeX2},
	    pdfcreator={LaTeX with abnTeX2},
		pdfkeywords={abnt}{latex}{abntex}{abntex2}{artigo científico},
		colorlinks=true,       		% false: boxed links; true: colored links
    	linkcolor=blue,          	% color of internal links
    	citecolor=blue,        		% color of links to bibliography
    	filecolor=magenta,      		% color of file links
		urlcolor=blue,
		bookmarksdepth=4
}
\makeatother
% ---

% ---
% compila o indice
% ---
\makeindex
% ---

% ---
% Altera as margens padrões
% ---
\setlrmarginsandblock{3cm}{3cm}{*}
\setulmarginsandblock{3cm}{3cm}{*}
\checkandfixthelayout
% ---

% ---
% Espaçamentos entre linhas e parágrafos
% ---

% O tamanho do parágrafo é dado por:
\setlength{\parindent}{1.3cm}

% Controle do espaçamento entre um parágrafo e outro:
\setlength{\parskip}{0.2cm}  % tente também \onelineskip

% Espaçamento simples
\SingleSpacing

% ----
% Início do documento
% ----
\begin{document}

% Seleciona o idioma do documento (conforme pacotes do babel)
%\selectlanguage{english}
\selectlanguage{brazil}

% Retira espaço extra obsoleto entre as frases.
\frenchspacing

% ----------------------------------------------------------
% ELEMENTOS PRÉ-TEXTUAIS
% ----------------------------------------------------------

%---
%
% Se desejar escrever o artigo em duas colunas, descomente a linha abaixo
% e a linha com o texto ``FIM DE ARTIGO EM DUAS COLUNAS''.
% \twocolumn[    		% INICIO DE ARTIGO EM DUAS COLUNAS
%
%---
% página de titulo
\maketitle

% resumo em português
\begin{resumoumacoluna}
    Este trabalho apresenta um estudo sobre dispositivos móveis, explicando a
    implementação de um protótipo de software que auxiliará no recebimento e
    análise de denuncias relacionados a iluminação pública na cidade de
    São José dos Campos. O aplicativo foi desenvolvido em Android e será
    responsável por enviar estas denuncias para um ambiente web.
    A aplicação web será responsável por responder às requisições do aplicativo
    Android e processa-las de modo que a manutenção dos pontos de iluminação
    pública seja feita de uma maneira mais rápida e menos burocrática.
 \vspace{\onelineskip}

 \noindent
 \textbf{Palavras-chave}: Iluminação pública. Aplicativo Android. Sistema web para manutenção.
\end{resumoumacoluna}

% ]  				% FIM DE ARTIGO EM DUAS COLUNAS
% ---

% ----------------------------------------------------------
% ELEMENTOS TEXTUAIS
% ----------------------------------------------------------
\textual

% ----------------------------------------------------------
% Introdução
% ----------------------------------------------------------
\section*{Introdução}
\addcontentsline{toc}{section}{Introdução}

A Iluminação pública é um instrumento de cidadania, pois é um item essencial
à qualidade de vida e a segurança dos centros urbanos, ajudando a preservar e
embelezar os patrimônios públicos e a prática de diversas atividade noturnas
no espaço público.

Segundo \cite{aver} a falta de iluminação contribui consideravelmente com o aumento
dos índices de criminalidade e a falta de segurança dos cidadãos que por algum
motivo tem que transitar no período noturno por pontos sem iluminação, ou com
iluminação precária. Estudos feitos pela organização Campbell Collaboration \cite{newton},
mostraram que a melhoria na iluminação pública em algumas cidades do Reino Unido e EUA, ajudaram a reduzir a criminalidade em até 21%, e concluiu que esta melhoria promoveu o aumento da vigilância da comunidade, que atuou como elemento inibidor contra a criminalidade local.
Os responsáveis pela instalação e manutenção da iluminação pública dos municípios
no Brasil mudou em dezembro de 2014. Até então, elas eram feitas pelo estado,
normalmente através das concessionárias de energia elétrica. Agora é a própria
prefeitura que tem a responsabilidade por garantir as lâmpadas acessas. Essa
mudança foi ocasionada pela Resolução Normativa 414 \cite{aneel414}, uma determinação
constitucional expedida pela ANEEL (Agencia Nacional de Energia Elétrica) que
obrigou todos os 5.570 municípios brasileiros a se responsabilizar pela
manutenção da infraestrutura da iluminação pública.

A cidade de São José dos Campos/SP conta hoje com cerca de 75.000 pontos de iluminação
distribuídos pela cidade. A média de lâmpadas com defeito é cerca de 900 por mês, de
acordo com a Secretaria de Obras da cidade \cite{secretariaobras}. Um dos maiores problemas a
serem enfrentados é o recebimento de chamados para solucionar problemas com os
postes defeituosos, já que o canal de reclamações da prefeitura engloba todos
os problemas da cidade, causando assim, uma sobrecarga no atendimento.

A Urbanizadora Municipal (URBAM) foi a empresa contratada pela prefeitura de
São José dos Campos para realizar a manutenção da iluminação pública do município.
Com a proposta de auxiliar a URBAM no encontro dos pontos com problemas e agilizar
as respostas aos chamados abertos pelos munícipes, o trabalho “Cidade Iluminada”
apresenta uma solução de um aplicativo para smartphones e uma aplicação Web para
a gestão das solicitações de manutenção dos pontos de iluminação pública.

O atual processo de recebimento de chamados, funciona de maneira manual, onde
um funcionário deve entrar em um sistema do canal de recebimentos de denuncias,
o 156, e coletar manualmente todas as denuncias recebidas no dia, colocar em uma
planilha no Excel, e separar por regiões. Após este passo, a planilha já
dividida em regiões, são checados os chamados um a um caso haja duplicidade.
Uma vez que todos os pedidos já estão na planilha e caso não haja duplicidade,
uma ordem de serviço é feita e enviada a URBAM para que a manutenção seja feita.
Quando a manutenção é feita, esta ordem de serviço retorna e são retirados da
planilha os chamados que foram atendidos.

Evidentemente é um processo muito complexo, e passível a erros, onde por um
menor descuido vários detalhes poderão ser perdidos, como denuncias duplicadas
que irão gerar o retrabalho quando a ordem de serviço é gerada. Por isso nós
propomos uma solução, onde não será mais necessária a coleta e verificação
manual de todas as denuncias. Todo este trabalho excessivo será reduzido e em
poucos cliques toda a análise será feita automaticamente.

Nossa missão é disponibilizar uma ferramenta que irá aperfeiçoar o recebimento
dos chamados relacionados aos postes de iluminação e viabilizar a comunicação
do cidadão diretamente com a entidade competente  através de uma aplicação
Android e um ambiente Web onde essas informações serão processadas . Neste
aplicativo, o usuário poderá registrar o problema na rede pública de iluminação,
enviar uma foto e a localização do problema e acompanhar o andamento do
protocolo aberto por ele. No ambiente Web, este protocolo será analisado e
separado de acordo com a região, bairro, rua e tipo de problemas, também serão
analisados se não há duplicidade em algum ponto. E além disso, o sistema tem
como função preparar a ordem de serviço para a manutenção, e fazer o controle
dos materiais utilizados.

% ----------------------------------------------------------
% Seção de explicações
% ----------------------------------------------------------
\section{Fundamentação Teórica}

\subsection{Engenharia de Software}

Segundo Pressman, a engenharia de software é definida como: “O estabelecimento
e uso de sólidos princípios de engenharia para que se possa obter economicamente
um software que seja confiável e que funcione eficientemente em máquinas reais.”\cite{pressman}

O termo engenharia de software foi criado por volta da década de 60, com o
objetivo de sanar a crise do software e contornar os problemas oriundos do
grande crescimento da demanda de softwares e da complexidade a eles imposta,
assim padronizando o desenvolvimento deixando-o mais controlado e sistemático.

Segundo Breitman “o desenvolvimento de sistemas de software é uma
atividade complexa que envolve um grande número de recursos, coordenados de
modo a atingir um mesmo objetivo”. \cite{breitman} Por isso esta tarefa exige um planejamento
detalhado e documentado por se tratar de um objeto dinâmico que está sempre se
adaptando as necessidades dos usuários de acordo com as ferramentas usadas no
seu desenvolvimento

Para Pressman, a engenharia de software tem três fases: definição,
desenvolvimento e manutenção \cite{pressman}.

A fase de definição, estabelece as propriedades, características, limitações e
funcionalidades do software. Esta fase é dividida em três etapas, a análise do
sistema onde são definidas as funções que o sistema deve apresentar.
O planejamento de projeto, onde  são analisados os riscos, custos e recursos que
serão necessários no desenvolvimento. E por fim a análise dos requisitos,
nesta etapa são verificados os requisitos necessários para que atenda as
funções do sistema.

A fase de desenvolvimento, é a parte onde os desenvolvedores transformas as
especificações em código. O projeto é analisado, estabelecendo a arquitetura e
estrutura dos dados, seguida pela codificação numa linguagem de programação
que atenta as necessidades do projeto.

E finalmente a fase de manutenção dedica-se a aprimorar ainda amais o sistema,
sendo validado pelo usuário final, que nem sempre está satisfeito. Algumas
mudanças podem ocorrer no projeto final, como correções pontuais que acontecem
quando usuário acha um erro, adaptações caso novos atributos precisarem ser
adicionados, e por fim o aprimoramento funcional, que acontece a medida que o
usuário se acostuma a utilizar o sistema, e poderá sugerir novas funcionalidades.

\subsection{Método de desenvolvimento Agile, Kanban e Scrum}

O Agile foi publicado em 2001 por meio da publicação de um manifesto por Jeff
Sutherland, Ken Schwaber et. al. que priorizava a rápida adaptação às novas
situações, a colaboração entre si dos times de desenvolvimento, a colaboração dos
times de desenvolvimento com os seus clientes e a entrega de softwares que estejam
funcionando. [5]

Apesar do Agile não ditar como o desenvolvimento deva ser feito por si só, ele
é um conjunto de conceitos para que métodos, como o Kanban e o Scrum, sejam
construídos.

O Scrum consiste basicamente em separar atividades de trabalho em sprints, com
o objetivo de entregar todas as atividades da forma mais completa possível,
trabalhando assim de modo incremental a cada sprint até que o cliente decida que
o software esteja com uma qualidade suficiente para ser entregue. Além disso,
seguindo os princípios dos métodos Agile, o Scrum ajuda na colaboração intra-equipe
por meio de reuniões stand-up diárias, uma reunião pós-sprint de retrospectiva
e uma reunião pré-sprint para planejamento do próximo sprint. Para facilitar o
planejamento, existem somente três “papéis” no Scrum: O Dono do Produto (Product
Owner), o Mestre do Scrum (Scrum Master) e a Equipe (Team).

A Equipe consiste nos desenvolvedores que irão efetivamente fazer a
implementação do produto baseado nas necessidades do Dono, que é responsável por
demonstrar os seus requisitos. O Mestre do Scrum é responsável por além de
transformar os requisitos do Dono em atividades que podem ser divididas entre os
membros da Equipe, é responsável principalmente por gerenciar a equipe para que
as atividades sejam cumpridas em cada sprint. [6]

O Kanban é uma técnica que separa as atividades em estados pré-definidos pela
equipe que a utiliza, e é utilizada principalmente para visualizar o progresso
de diferentes atividades dentro de um contexto. Por si só, o Kanban não diz como
introduzir novos processos, mas sim como montar os processos existentes, como o
Scrum, de forma visual.

A norma ABNT NBR 6022:2003 não estabelece uma margem específica a ser utilizada
no artigo científico. Dessa maneira, caso deseje alterar as margens, utilize os
comandos abaixo:

\begin{verbatim}
   \setlrmarginsandblock{3cm}{3cm}{*}
   \setulmarginsandblock{3cm}{3cm}{*}
   \checkandfixthelayout
\end{verbatim}

\subsection{Duas colunas}

É comum que artigos científicos sejam escritos em duas colunas. Para isso,
adicione a opção \texttt{twocolumn} à classe do documento, como no exemplo:

\begin{verbatim}
   \documentclass[article,11pt,oneside,a4paper,twocolumn]{abntex2}
\end{verbatim}

É possível indicar pontos do texto que se deseja manter em apenas uma coluna,
geralmente o título e os resumos. Os resumos em única coluna em documentos com
a opção \texttt{twocolumn} devem ser escritos no ambiente
\texttt{resumoumacoluna}:

\begin{verbatim}
   \twocolumn[              % INICIO DE ARTIGO EM DUAS COLUNAS

     \maketitle             % pagina de titulo

     \renewcommand{\resumoname}{Nome do resumo}
     \begin{resumoumacoluna}
        Texto do resumo.

        \vspace{\onelineskip}

        \noindent
        \textbf{Palavras-chave}: latex. abntex. editoração de texto.
     \end{resumoumacoluna}

   ]                        % FIM DE ARTIGO EM DUAS COLUNAS
\end{verbatim}

\subsection{Recuo do ambiente \texttt{citacao}}

Na produção de artigos (opção \texttt{article}), pode ser útil alterar o recuo
do ambiente \texttt{citacao}. Nesse caso, utilize o comando:

\begin{verbatim}
   \setlength{\ABNTEXcitacaorecuo}{1.8cm}
\end{verbatim}

Quando um documento é produzido com a opção \texttt{twocolumn}, a classe
\textsf{abntex2} automaticamente altera o recuo padrão de 4 cm, definido pela
ABNT NBR 10520:2002 seção 5.3, para 1.8 cm.

\section{Cabeçalhos e rodapés customizados}

Diferentes estilos de cabeçalhos e rodapés podem ser criados usando os
recursos padrões do \textsf{memoir}.

Um estilo próprio de cabeçalhos e rodapés pode ser diferente para páginas pares
e ímpares. Observe que a diferenciação entre páginas pares e ímpares só é
utilizada se a opção \texttt{twoside} da classe \textsf{abntex2} for utilizado.
Caso contrário, apenas o cabeçalho padrão da página par (\emph{even}) é usado.

Veja o exemplo abaixo cria um estilo chamado \texttt{meuestilo}. O código deve
ser inserido no preâmbulo do documento.

\begin{verbatim}
%%criar um novo estilo de cabeçalhos e rodapés
\makepagestyle{meuestilo}
  %%cabeçalhos
  \makeevenhead{meuestilo} %%pagina par
     {topo par à esquerda}
     {centro \thepage}
     {direita}
  \makeoddhead{meuestilo} %%pagina ímpar ou com oneside
     {topo ímpar/oneside à esquerda}
     {centro\thepage}
     {direita}
  \makeheadrule{meuestilo}{\textwidth}{\normalrulethickness} %linha
  %% rodapé
  \makeevenfoot{meuestilo}
     {rodapé par à esquerda} %%pagina par
     {centro \thepage}
     {direita}
  \makeoddfoot{meuestilo} %%pagina ímpar ou com oneside
     {rodapé ímpar/onside à esquerda}
     {centro \thepage}
     {direita}
\end{verbatim}

Para usar o estilo criado, use o comando abaixo imediatamente após um dos
comandos de divisão do documento. Por exemplo:

\begin{verbatim}
   \begin{document}
     %%usar o estilo criado na primeira página do artigo:
     \pretextual
     \pagestyle{meuestilo}

     \maketitle
     ...

     %%usar o estilo criado nas páginas textuais
     \textual
     \pagestyle{meuestilo}

     \chapter{Novo capítulo}
     ...
   \end{document}
\end{verbatim}

Outras informações sobre cabeçalhos e rodapés estão disponíveis na seção 7.3 do
manual do \textsf{memoir} \cite{memoir}.

\section{Mais exemplos no Modelo Canônico de Trabalhos Acadêmicos}

Este modelo de artigo é limitado em número de exemplos de comandos, pois são
apresentados exclusivamente comandos diretamente relacionados com a produção de
artigos.

Para exemplos adicionais de \abnTeX\ e \LaTeX, como inclusão de figuras,
fórmulas matemáticas, citações, e outros, consulte o documento
\citeonline{abntex2modelo}.

\section{Consulte o manual da classe \textsf{abntex2}}

Consulte o manual da classe \textsf{abntex2} \cite{abntex2classe} para uma
referência completa das macros e ambientes disponíveis.

% ---
% Finaliza a parte no bookmark do PDF, para que se inicie o bookmark na raiz
% ---
\bookmarksetup{startatroot}%
% ---

% ---
% Conclusão
% ---
\section*{Considerações finais}
\addcontentsline{toc}{section}{Considerações finais}

\lipsum[1]

\begin{citacao}
\lipsum[2]
\end{citacao}

\lipsum[3]

% ----------------------------------------------------------
% ELEMENTOS PÓS-TEXTUAIS
% ----------------------------------------------------------
\postextual

% ---
% Título e resumo em língua estrangeira
% ---

% \twocolumn[    		% INICIO DE ARTIGO EM DUAS COLUNAS

% titulo em inglês
\titulo{Canonical academic article model with \abnTeX}
\emptythanks
\maketitle

% resumo em português
\renewcommand{\resumoname}{Abstract}
\begin{resumoumacoluna}
 \begin{otherlanguage*}{english}
   This paper presents a study on mobile development, explaining the
   implementation of a software prototype that will assist in receiving
   complaints and analysis related to street lighting in the city of São José
   dos Campos. The application was developed on Android and will be responsible
   for sending these complaints to a web environment. The web application
   will be responsible for responding to requests from the Android application
   and process them so that the maintenance of public lighting points is done
   in a faster and less bureaucratic way.
   \vspace{\onelineskip}

   \noindent
   \textbf{Keywords}: Public lighting system. Android Application. Web system for maintenance.
 \end{otherlanguage*}
\end{resumoumacoluna}

% ]  				% FIM DE ARTIGO EM DUAS COLUNAS
% ---

% ----------------------------------------------------------
% Referências bibliográficas
% ----------------------------------------------------------
\bibliography{bibliografia}

% ----------------------------------------------------------
% Glossário
% ----------------------------------------------------------
%
% Há diversas soluções prontas para glossário em LaTeX.
% Consulte o manual do abnTeX2 para obter sugestões.
%
%\glossary

% ----------------------------------------------------------
% Apêndices
% ----------------------------------------------------------

% ---
% Inicia os apêndices
% ---
\begin{apendicesenv}

% ----------------------------------------------------------
\chapter{Nullam elementum urna vel imperdiet sodales elit ipsum pharetra ligula
ac pretium ante justo a nulla curabitur tristique arcu eu metus}
% ----------------------------------------------------------
\lipsum[55-57]

\end{apendicesenv}
% ---

% ----------------------------------------------------------
% Anexos
% ----------------------------------------------------------
\cftinserthook{toc}{AAA}
% ---
% Inicia os anexos
% ---
%\anexos
\begin{anexosenv}

% ---
\chapter{Cras non urna sed feugiat cum sociis natoque penatibus et magnis dis
parturient montes nascetur ridiculus mus}
% ---

\lipsum[31]

\end{anexosenv}

\end{document}
